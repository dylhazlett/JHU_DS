% Options for packages loaded elsewhere
\PassOptionsToPackage{unicode}{hyperref}
\PassOptionsToPackage{hyphens}{url}
%
\documentclass[
]{article}
\usepackage{amsmath,amssymb}
\usepackage{lmodern}
\usepackage{ifxetex,ifluatex}
\ifnum 0\ifxetex 1\fi\ifluatex 1\fi=0 % if pdftex
  \usepackage[T1]{fontenc}
  \usepackage[utf8]{inputenc}
  \usepackage{textcomp} % provide euro and other symbols
\else % if luatex or xetex
  \usepackage{unicode-math}
  \defaultfontfeatures{Scale=MatchLowercase}
  \defaultfontfeatures[\rmfamily]{Ligatures=TeX,Scale=1}
\fi
% Use upquote if available, for straight quotes in verbatim environments
\IfFileExists{upquote.sty}{\usepackage{upquote}}{}
\IfFileExists{microtype.sty}{% use microtype if available
  \usepackage[]{microtype}
  \UseMicrotypeSet[protrusion]{basicmath} % disable protrusion for tt fonts
}{}
\makeatletter
\@ifundefined{KOMAClassName}{% if non-KOMA class
  \IfFileExists{parskip.sty}{%
    \usepackage{parskip}
  }{% else
    \setlength{\parindent}{0pt}
    \setlength{\parskip}{6pt plus 2pt minus 1pt}}
}{% if KOMA class
  \KOMAoptions{parskip=half}}
\makeatother
\usepackage{xcolor}
\IfFileExists{xurl.sty}{\usepackage{xurl}}{} % add URL line breaks if available
\IfFileExists{bookmark.sty}{\usepackage{bookmark}}{\usepackage{hyperref}}
\hypersetup{
  pdftitle={Statistical Inference Course Project},
  pdfauthor={Dylan Hazlett},
  hidelinks,
  pdfcreator={LaTeX via pandoc}}
\urlstyle{same} % disable monospaced font for URLs
\usepackage[margin=1in]{geometry}
\usepackage{color}
\usepackage{fancyvrb}
\newcommand{\VerbBar}{|}
\newcommand{\VERB}{\Verb[commandchars=\\\{\}]}
\DefineVerbatimEnvironment{Highlighting}{Verbatim}{commandchars=\\\{\}}
% Add ',fontsize=\small' for more characters per line
\usepackage{framed}
\definecolor{shadecolor}{RGB}{248,248,248}
\newenvironment{Shaded}{\begin{snugshade}}{\end{snugshade}}
\newcommand{\AlertTok}[1]{\textcolor[rgb]{0.94,0.16,0.16}{#1}}
\newcommand{\AnnotationTok}[1]{\textcolor[rgb]{0.56,0.35,0.01}{\textbf{\textit{#1}}}}
\newcommand{\AttributeTok}[1]{\textcolor[rgb]{0.77,0.63,0.00}{#1}}
\newcommand{\BaseNTok}[1]{\textcolor[rgb]{0.00,0.00,0.81}{#1}}
\newcommand{\BuiltInTok}[1]{#1}
\newcommand{\CharTok}[1]{\textcolor[rgb]{0.31,0.60,0.02}{#1}}
\newcommand{\CommentTok}[1]{\textcolor[rgb]{0.56,0.35,0.01}{\textit{#1}}}
\newcommand{\CommentVarTok}[1]{\textcolor[rgb]{0.56,0.35,0.01}{\textbf{\textit{#1}}}}
\newcommand{\ConstantTok}[1]{\textcolor[rgb]{0.00,0.00,0.00}{#1}}
\newcommand{\ControlFlowTok}[1]{\textcolor[rgb]{0.13,0.29,0.53}{\textbf{#1}}}
\newcommand{\DataTypeTok}[1]{\textcolor[rgb]{0.13,0.29,0.53}{#1}}
\newcommand{\DecValTok}[1]{\textcolor[rgb]{0.00,0.00,0.81}{#1}}
\newcommand{\DocumentationTok}[1]{\textcolor[rgb]{0.56,0.35,0.01}{\textbf{\textit{#1}}}}
\newcommand{\ErrorTok}[1]{\textcolor[rgb]{0.64,0.00,0.00}{\textbf{#1}}}
\newcommand{\ExtensionTok}[1]{#1}
\newcommand{\FloatTok}[1]{\textcolor[rgb]{0.00,0.00,0.81}{#1}}
\newcommand{\FunctionTok}[1]{\textcolor[rgb]{0.00,0.00,0.00}{#1}}
\newcommand{\ImportTok}[1]{#1}
\newcommand{\InformationTok}[1]{\textcolor[rgb]{0.56,0.35,0.01}{\textbf{\textit{#1}}}}
\newcommand{\KeywordTok}[1]{\textcolor[rgb]{0.13,0.29,0.53}{\textbf{#1}}}
\newcommand{\NormalTok}[1]{#1}
\newcommand{\OperatorTok}[1]{\textcolor[rgb]{0.81,0.36,0.00}{\textbf{#1}}}
\newcommand{\OtherTok}[1]{\textcolor[rgb]{0.56,0.35,0.01}{#1}}
\newcommand{\PreprocessorTok}[1]{\textcolor[rgb]{0.56,0.35,0.01}{\textit{#1}}}
\newcommand{\RegionMarkerTok}[1]{#1}
\newcommand{\SpecialCharTok}[1]{\textcolor[rgb]{0.00,0.00,0.00}{#1}}
\newcommand{\SpecialStringTok}[1]{\textcolor[rgb]{0.31,0.60,0.02}{#1}}
\newcommand{\StringTok}[1]{\textcolor[rgb]{0.31,0.60,0.02}{#1}}
\newcommand{\VariableTok}[1]{\textcolor[rgb]{0.00,0.00,0.00}{#1}}
\newcommand{\VerbatimStringTok}[1]{\textcolor[rgb]{0.31,0.60,0.02}{#1}}
\newcommand{\WarningTok}[1]{\textcolor[rgb]{0.56,0.35,0.01}{\textbf{\textit{#1}}}}
\usepackage{graphicx}
\makeatletter
\def\maxwidth{\ifdim\Gin@nat@width>\linewidth\linewidth\else\Gin@nat@width\fi}
\def\maxheight{\ifdim\Gin@nat@height>\textheight\textheight\else\Gin@nat@height\fi}
\makeatother
% Scale images if necessary, so that they will not overflow the page
% margins by default, and it is still possible to overwrite the defaults
% using explicit options in \includegraphics[width, height, ...]{}
\setkeys{Gin}{width=\maxwidth,height=\maxheight,keepaspectratio}
% Set default figure placement to htbp
\makeatletter
\def\fps@figure{htbp}
\makeatother
\setlength{\emergencystretch}{3em} % prevent overfull lines
\providecommand{\tightlist}{%
  \setlength{\itemsep}{0pt}\setlength{\parskip}{0pt}}
\setcounter{secnumdepth}{-\maxdimen} % remove section numbering
\ifluatex
  \usepackage{selnolig}  % disable illegal ligatures
\fi

\title{Statistical Inference Course Project}
\author{Dylan Hazlett}
\date{}

\begin{document}
\maketitle

Thank you for reviewing my final report for the Statistical Inference
Course. In Part 1, we will simulate a collection exponential variables,
then compare the sample's summary statistics to their theoretical values
and distribution. In Part 2, we will analyze the differences in
tooth-growth among varying doses and delivery methods of Vitamin C using
the ToothGrowth dataset in R.

\hypertarget{part-1-simulation-of-expontentially-distributed-variables}{%
\subsubsection{Part 1: Simulation of Expontentially Distributed
Variables}\label{part-1-simulation-of-expontentially-distributed-variables}}

\begin{Shaded}
\begin{Highlighting}[]
\NormalTok{mns }\OtherTok{=} \ConstantTok{NULL}
\NormalTok{vars }\OtherTok{=} \ConstantTok{NULL}
\ControlFlowTok{for}\NormalTok{ (i }\ControlFlowTok{in} \DecValTok{1} \SpecialCharTok{:} \DecValTok{1000}\NormalTok{) \{}
\NormalTok{  mns }\OtherTok{=} \FunctionTok{c}\NormalTok{(mns, }\FunctionTok{mean}\NormalTok{(}\FunctionTok{rexp}\NormalTok{(}\DecValTok{40}\NormalTok{, .}\DecValTok{2}\NormalTok{))) }
\NormalTok{\}}
\end{Highlighting}
\end{Shaded}

The above block of code simulates the distribution of averages of 40
exponentials with lambda value .2, 1000 times. After initializing `mns'
to record the means, a for loop runs the exponential distribution
expressions and gathers each mean 1000 times. The output is an array of
1000 independently sampled means.

\begin{verbatim}
## [1] "Using the mean() function on the array, mns, we can calculate the average Sample Mean: 5.02528893363976"
\end{verbatim}

Theoretically, and with enough simulations, this value will converage to
the mean of the exponential distribution, 1/lambda or 5 since lambda =
.2.
\includegraphics{Statistical-Inference_files/figure-latex/unnamed-chunk-3-1.pdf}
This figure shows the distribution of the 1000 sample means with two
vertical lines, the average sample mean in blue and the theoretical mean
in red. The two may appear to be touching on the chart due to how close
they are in value.

\begin{verbatim}
## [1] "Using the var() function, we can calculate that the Sample Variance is 0.651939499915466"
\end{verbatim}

To compare the sample variance to its theoretical value for the
exponential distribution, we use the following formula given that n
equals 40 and sigma equals the standard deviation, given that the
standard deviation of the exponential distribution is 1/lambda where
lambda equals 0.2. \(\bar Var = \sigma^2 / n\)

\begin{verbatim}
## [1] "The Theoretical Variance for 40 exponential variables where lambda equal 0.2 is 0.625"
\end{verbatim}

Given infinite simulations, the sample variance will converge to this
value of 0.625.

This simulation is an example of the Central Limit Theorem, where the
mean samples resulting from the simulations are normally distributed
around the theoretical mean.
\includegraphics{Statistical-Inference_files/figure-latex/unnamed-chunk-6-1.pdf}
The exponential distribution is of course not normally distributed; but,
after taking 1000 random and independent samples of the mean of this
distrubtion, we get the below distribution that appears to be a normal
Guassian distribution of samples.
\includegraphics{Statistical-Inference_files/figure-latex/unnamed-chunk-7-1.pdf}
Although the right tail of this distribution stretches further from the
mean than that of the right, that is expected due to the nature of the
exponential distribution stretching to infinity on the x-axis. We accept
this drawback and conclude that the distribution is approximately
normal. Using the central limit theorem, we assume iid, meaning our
exponential variables were sampled randomly and independently of one
another. We also assume that the sample size of 40 independent variable
was sufficiently large, which is a safe assumption due to the resulting
distribution chart.

\hypertarget{part-2-toothgrowth}{%
\subsection{Part 2: ToothGrowth}\label{part-2-toothgrowth}}

Now, for Part 2, lets load and explore the ToothGrowth dataset in R.

\begin{Shaded}
\begin{Highlighting}[]
\FunctionTok{library}\NormalTok{(dplyr)}
\end{Highlighting}
\end{Shaded}

\begin{verbatim}
## 
## Attaching package: 'dplyr'
\end{verbatim}

\begin{verbatim}
## The following objects are masked from 'package:stats':
## 
##     filter, lag
\end{verbatim}

\begin{verbatim}
## The following objects are masked from 'package:base':
## 
##     intersect, setdiff, setequal, union
\end{verbatim}

\begin{Shaded}
\begin{Highlighting}[]
\NormalTok{ToothGrowth}\SpecialCharTok{\%\textgreater{}\%} \FunctionTok{group\_by}\NormalTok{(dose, supp)}\SpecialCharTok{\%\textgreater{}\%}\FunctionTok{summarise}\NormalTok{(}\FunctionTok{mean}\NormalTok{(len), }\FunctionTok{n}\NormalTok{())}
\end{Highlighting}
\end{Shaded}

\begin{verbatim}
## `summarise()` has grouped output by 'dose'. You can override using the `.groups` argument.
\end{verbatim}

\begin{verbatim}
## # A tibble: 6 x 4
## # Groups:   dose [3]
##    dose supp  `mean(len)` `n()`
##   <dbl> <fct>       <dbl> <int>
## 1   0.5 OJ          13.2     10
## 2   0.5 VC           7.98    10
## 3   1   OJ          22.7     10
## 4   1   VC          16.8     10
## 5   2   OJ          26.1     10
## 6   2   VC          26.1     10
\end{verbatim}

The ToothGrowth dataset has 60 data entries of tooth length (len),
dosage mg/day (dos), and supplement type (supp). With two supplements
(VC or OJ) and 3 dosage levels (0.5, 1, and 2 mg/day), there are 10 data
entries to each of the 6 group combinations of dosage and suplement.
Since these entries are not paired, any comparison between these groups
will use the mean, which is provided to give a rough expectation of what
groups may have significantly different lengths. It apprears that lower
doses contribute to less tooth lenght growth, as well as more growth
with the OJ supplement vs the VC supplement for lower doses.
\includegraphics{Statistical-Inference_files/figure-latex/unnamed-chunk-8-1.pdf}
Here is a visual comparison of all of the length data points split by
dosage on the x-axis and supplement for color. We can see that, at a
dosage of 2 mg/day, there is little difference in the distribution of
length among supplements. We will keep this and our earlier hypothesis
in mind as we now construct confidence intervals for all 6 groups.

\begin{Shaded}
\begin{Highlighting}[]
\NormalTok{ToothGrowth}\SpecialCharTok{\%\textgreater{}\%} \FunctionTok{group\_by}\NormalTok{(dose, supp)}\SpecialCharTok{\%\textgreater{}\%}\FunctionTok{summarise}\NormalTok{(}\FunctionTok{mean}\NormalTok{(len),}\AttributeTok{lower\_int =} \FunctionTok{mean}\NormalTok{(len)}\SpecialCharTok{{-}}\NormalTok{(}\FunctionTok{qt}\NormalTok{(.}\DecValTok{975}\NormalTok{,}\DecValTok{9}\NormalTok{)}\SpecialCharTok{*}\FunctionTok{sd}\NormalTok{(len)}\SpecialCharTok{/} \FunctionTok{sqrt}\NormalTok{(}\FunctionTok{n}\NormalTok{())), }\AttributeTok{upper\_int =} \FunctionTok{mean}\NormalTok{(len)}\SpecialCharTok{+}\NormalTok{(}\FunctionTok{qt}\NormalTok{(.}\DecValTok{975}\NormalTok{,}\DecValTok{9}\NormalTok{)}\SpecialCharTok{*}\FunctionTok{sd}\NormalTok{(len)}\SpecialCharTok{/} \FunctionTok{sqrt}\NormalTok{(}\FunctionTok{n}\NormalTok{())))}
\end{Highlighting}
\end{Shaded}

\begin{verbatim}
## `summarise()` has grouped output by 'dose'. You can override using the `.groups` argument.
\end{verbatim}

\begin{verbatim}
## # A tibble: 6 x 5
## # Groups:   dose [3]
##    dose supp  `mean(len)` lower_int upper_int
##   <dbl> <fct>       <dbl>     <dbl>     <dbl>
## 1   0.5 OJ          13.2      10.0      16.4 
## 2   0.5 VC           7.98      6.02      9.94
## 3   1   OJ          22.7      19.9      25.5 
## 4   1   VC          16.8      15.0      18.6 
## 5   2   OJ          26.1      24.2      28.0 
## 6   2   VC          26.1      22.7      29.6
\end{verbatim}

The figure above shows the 95\% t confidence interval for each of the 6
testing groups. We can conclude that for any of the intervals that have
overlapping ranges, a t test will not have a significant result at 95\%
confidence - these comparisons include: OJ vs VC at 2.0 dosage, OJ at
1.0 dosage compared to either supplement groups at 2.0 dosage, OJ at 0.5
dosage vs VC at 1.0 dosage. We can make several significant conclusions
about the individual group comparisons that do not intersect eachother's
t confidence interval, but it may be more useful to go over the higher
level implication with this dataset. It would appear that, at lower
doses (0.5 and 1.0), the OJ supplement has a significant greater effect
on tooth growth than that of the VC supplement; however, at the highest
dosage of 2.0, the tooth growth differences between supplements are
negligable. Though tooth growth for the OJ groups does increase as the
dosage increases (we are more than 95\% confident that there is a length
difference between OJ groups of 0.5 mg/day and 2.0mg/day), lengths at
consecutive doses are not different at 95\% confidence; whereas VC
supplement groups show steep, significant differences between all dosage
levels. Because the data groups here are small, 10 subjects in each, we
must be cautious with definitive conclusions about the population.

Lastly, let's run a t test between supplements on all subjects with
equal variances set to False: Given OJ, Var(len) = 43.63 Given VC,
Var(len) = 68.33 Note: This test cannot be preformed with dosage as
there are more than two groups.

\begin{Shaded}
\begin{Highlighting}[]
\FunctionTok{t.test}\NormalTok{(len }\SpecialCharTok{\textasciitilde{}}\NormalTok{ supp, }\AttributeTok{paired =} \ConstantTok{FALSE}\NormalTok{, }\AttributeTok{var.equal =} \ConstantTok{FALSE}\NormalTok{, }\AttributeTok{data =}\NormalTok{ ToothGrowth)}
\end{Highlighting}
\end{Shaded}

\begin{verbatim}
## 
##  Welch Two Sample t-test
## 
## data:  len by supp
## t = 1.9153, df = 55.309, p-value = 0.06063
## alternative hypothesis: true difference in means between group OJ and group VC is not equal to 0
## 95 percent confidence interval:
##  -0.1710156  7.5710156
## sample estimates:
## mean in group OJ mean in group VC 
##         20.66333         16.96333
\end{verbatim}

Using the t test result, we cannot conclude with 95\% confidence that
tooth growth is different between supplement groups.

To use the t confidence intervals and the t test we assumes that the
data are iid and have a roughly symmetrical distribution.

\end{document}
